\colorlet{chaptergrey}{blue!20!}
\chapter[Kinematics, the cosmic web, and halo assembly]{Exploring the role of the cosmic web and halo assembly in galaxy evolution using kinematics}
\label{ch:halo_assembly}
\vspace{-5.25in}
\includegraphics[height=1.39in]{thesis/latex/misalignment_MaNGA/kin_mis_chapter_heading_grey.pdf}
\vspace{3in}

\epigraph{This chapter is partially based on Duckworth, Tojeiro, Kraljic, Sgr\'o, Wild, Weijmans, Lacerna and Drory, in MNRAS, 448, Issue 1, 2019 and Kraljic, Duckworth, Tojeiro et al. (in prep.). Here, we investigate how large scale IFS surveys can be utilized to understand the relationship between galaxy kinematics and their large-scale environment, in the context of dark matter halo assembly.}

\section{Introduction}
In Chapters 2 and 3, we demonstrated how galaxy kinematics are closely related to morphological properties, baryonic processes such as AGN feedback, and the characteristics of its surrounding dark matter halo. The stellar (dark matter halo) mass of a galaxy encodes a lot of its evolution, demonstrating that \textit{local environment} (and over-density) can explain a large part of the diversity of properties we observe.

As introduced in \S\ref{sec:cosmic_web_intro}, galaxies on large scales are organised within the cosmic network of filamentary structure. Dark matter haloes and their constituent galaxies are subject to the large-scale anisotropic forces throughout their formation and evolution, in particular shaping their initial angular momentum content, before non-linear baryonic processes take hold. There is growing evidence that the cosmic web environment of a galaxy, also plays a significant role in galaxy evolution and is, at least partially, driving their morphology. 

Revisiting tidal torque theory \citep[TTT; e.g.][]{hoyle1951, peebles1969} in the context of large-scale anisotropic environment, \citet{codis2015} explain the relative angular momentum distribution of haloes with respect to neighbouring filaments and walls. The misalignment between the inertia tensors of the proto-haloes and the directionality of the tidal tensor due to the neighbouring wall or filament leads to a spin alignment for low mass haloes. As these haloes grow in mass hierarchically, merging with other haloes and they lose this preferential alignment with large scale structure. Due to the flow of matter along walls and filaments, mergers often happen in this plane, possibly leading to a \textit{flip} in direction, so that higher mass dark matter haloes are preferentially perpendicular to the orientation of the tidal field \citep[e.g.][]{Codis2012, dubois2014, GaneshaiahVeena2018}. This understanding is motivated by cosmological N-body simulations and learning how this propagates to observations is complex. In simulations, the constituent galaxies appear to retain a memory of the spin orientation (with respect to large scale structure) as dictated by their host dark matter halo \citep[e.g.][]{codis2018, Kraljic2019flip}. The mass dependence of the spin alignment \textit{flip} is, however, debated and likely dependent on different choices for baryonic processes and the scale of the neighbouring filamentary structure.

The orientation of galaxies with respect to large-scale structure is, however, only one aspect in determining the relationship between galaxy evolution and the cosmic web. As introduced in \S\ref{sec:halo_assembly_bias_intro}, the large-scale tidal forces created by the cosmic web are proposed to modulate the assembly history of dark matter haloes and their constituent galaxies, once the effect of halo mass has been taken into account. Following the discussion of \citet{hahn2009}, the current ($z=0$) accretion rate onto dark matter haloes is correlated with the large-scale environment as quantified by the tidal field strength. The sustained ability for a halo to grow naturally modulates its formation time (as defined by the time at which the halo assembles half its current mass). This effect is often seen most clearly for low-mass haloes, who have their accretion `stalled' in the vicinity of large haloes or structures, leading to suppression in growth \citep[see also;][]{wang2007,dalal2008,lacerna2011}. The tidal fields due to the large scale environment decreases the size of the hill radius (i.e. where it can accrete from) for the low-mass haloes, effectively boosting the clustering of older low-mass haloes, compared to haloes of the same mass residing in regions of lesser tidal field strength. This can also be understood for a topological description of the cosmic web. In \citet{ZOMGI}, its demonstrated that low-mass haloes residing within (or in close vicinity to) large filamentary  structure (i.e. of width larger than the halo), can often see their accretion `stalled' due to the flow of matter along the filament after accretion. For this reason, material accreted perpendicular to the filamentary direction (i.e. coming in from walls) is seen to be on far more tangetial orbits (see \S\ref{sec:velocity_anisotropy} for more information) relative to the halo, leading to lower current accretion rates and hence earlier formation. Conversely low-mass haloes at the convergence point of multiple smaller filaments will have continued isotropic accretion resulting in longer continued mass growth and more recent formation. \citep[See][for a theoretical approach]{musso2018}.

Understanding how this propagates to galaxy observables is complex, however, an earlier forming halo would most likely lead to an earlier forming galaxy. In turn, a lower current dark matter accretion rate onto the halo could be mirrored by a lower rate of gas accretion onto the central galaxy, since the baryonic material is subject to the same tidal forces. In single fibre studies, one method of encapsulating the difference in assembly histories can be through the study of star formation histories. Under the assumption that the gas is cool enough to condense and form stars, continued gas accretion should constitute a near monotonic relationship with star formation, and hence, an earlier forming galaxy (or halo) could correspond to the `stalled' accretion. \red{needs some reference here.} One difficulty with single fibre studies is that the information obtained is an average across the galaxy, often only considering the brightest, central region. The exact influence of the cosmic web, particularly in the form of accretion is nuanced. Further, due to the potential alignment between galaxy spin and the neighbouring filamentary direction it seems unlikely that such accretion penetrates deep enough in the galaxy to be immediately observed in the central region of the galaxy. 

A powerful approach in a holistic understanding of galaxy evolution, and hence, potentially enabling us to trace the impact of dark matter halo assembly, is spatially resolved spectroscopy. As introduced in \S\ref{sec:ifs_surveys_intro} IFS surveys for thousands of galaxies, across a wide variety of properties, environments, and masses in the local Universe are now a reality. Current generation IFS surveys such as MaNGA and SAMI have opened the door for studies between detailed galaxy properties (e.g. kinematics or radial profiles of star formation) and environment, both local and large-scale. In this chapter, we use the MaNGA survey (MPL-6; $\sim 4600$ galaxies) to explore two questions:

\begin{itemize}
    \item Does the spin orientation of galaxy disks in the local Universe align with the local geometry of neighbouring filaments?
    \item Can we trace halo assembly (as measured by the current gas accretion rate) in galaxy kinematics? 
\end{itemize}

In \S\ref{sec:spin_alignment} we present the first study making use of galaxy kinematics to estimate the 3D alignment between the spin direction of galaxy disks and their neighbouring filamentary structure. We investigate the dependence of spin alignment on stellar mass and galaxy morphology, to better understand the connection between galaxy evolution and the cosmic web. In \S\ref{sec:halo_assembly} we investigate whether the dark matter halo assembly history can be related to the current rate of gas accretion onto central galaxies. We use kinematic misalignment as a proxy for ongoing assembly (i.e. gas accretion) and explore if this is related to different tracers of halo assembly (i.e. HOD modelling, vicinity to cosmic web morphological features, and substructure fraction). In each section we summarise our findings before concluding in \S\ref{sec:halo_assembly_conclusion}.

\section{Spin alignment of spiral and S0 galaxies in MaNGA} \label{sec:spin_alignment}
\red{add figure of spin orientation with disperse overlay?}
Historically, exploring the relationship between the spin direction of galaxies and large scale structure has been difficult. Without spatially resolved spectra of galaxies, we are reliant on projected shapes, which introduce degeneracies with respect to the actual spin vector direction \citep[e.g. see Fig 2. in][for example of degeneracies that can occur]{motloch2020}. Due to this, and the different approaches to re-constructing the cosmic web, studies are often conflicting in findings \citep[e.g. spiral galaxies having parallel vs perpendicular orientations with respect to the cosmic web][]{tempel2013a, tempel2013, lee2007, jones2010, zhang2015}. These differences due to the reconstruction of the cosmic web can likely be explained due a difference in chosen scales. In hydrodynamical simulations, the transition mass from preferential parallel alignment to perpendicular alignment is very much dependent on the scale of the filamentary structure, such as width \citep{ganeshaiahveena2019, Kraljic2019flip}. Averaging over an ensemble of filamentary scales will likely wash out this transition, and hence preferential alignment in either direction. 

The other half of the problem can be solved, however, by having kinematic information for a large population of galaxies. Kinematic (rotational) position angles help break the degeneracy of shape orientation, and provide more robust measures of directionality. We are now in the era of multiple IFS surveys that may observe enough galaxies to find a significant detection of spin alignment. In this section we explore the concept of spin alignment in MaNGA for spirals and S0 galaxies.

\subsection{Data and Methods}
\subsubsection{Galaxy sample}
Throughout this chapter, we base our analysis on 4633 unique galaxy observations corresponding to the 6th MaNGA product launch (MPL). This reflects the state of the survey as of mid-2018 (data released publicly December 2018). For each galaxy we compute kinematic position angles for both the stellar and ionized gas velocity fields. We calculate position angles as described in \S\ref{sec:kin_mis}. In this section, concerning spin alignment, we only make use of position angles derived from stellar kinematics, however, in \S\ref{sec:halo_assembly} we make use of the full kinematic misalignment measures. 

To compute the spin direction of each galaxy in our sample, we assume a thin disk approximation and therefore require each galaxy to have a disk component. We select our sample based on the visual morphological classifications of GalaxyZoo2 \citep[GZ2][]{willett2013} as introduced in \S\ref{sec:morph_def_obs}. We select a \textit{pure} spiral galaxy sample by taking all objects that have a debiased vote fraction of $\geq 0.9$ for answers positive to the galaxy having a disk. \red{add in lenticular definition}.

\subsubsection{DisPerSE}
For all work in this chapter, we characterise the topological features of the cosmic web using the Discrete Persistent Structure Extractor Code \citep{sousbie2011a, sousbie2011b} as introduced in \S\ref{sec:cosmic_web_intro}. Here DisPerSE is applied to a modified version of the SDSS DR10 spectroscopic sample as described in \citet{tempel2014}. In observations, there is an additional difficulty since we are not working with exact three dimensional positions in real space, but rather, in redshift space. To recover the intrinsic galaxy position, an estimation for the peculiar velocity must be made for each galaxy. On large-scales this corresponds to the Kaiser effect \citep{kaiser1987}, which acts to increase the contrast of the skeleton due the coherent motion of galaxies with the growth of structure \citep[e.g.][]{shi2016}. On small-scales, however, the Fingers of God effect \citep[FOG;][]{jackson1972,tulley1978} derives from random motions of galaxies within virialized haloes. The latter can elongate structure in redshift space leading to erroneous identification of filaments. We correct for the FOG effect using the technique outlined in \citet{kraljic2018}. As introduced in \S\ref{sec:cosmic_web_intro}, the \textit{robustness} of the cosmic web's morphological features identified are quantified using the so called persistence ratio, a measure of the significance of the topological connection between individual pairs of critical points. Here we use choose robust filaments identified with a significance of 5$\sigma$. In Figure \ref{fig:disperse_sdss} we show an example of the recovered density field from the set of galaxy positions in SDSS and the corresponding ridge extraction (i.e. filament detection). \red{length of typical segments with respect to typical distance to galaxy?}

\begin{figure*}
    \centering
	\includegraphics[width=\linewidth]{thesis/latex/halo_assembly_manga/SDSS_CW_DisPerSE.pdf}
    \caption[Illustration of the filamentary network for a slice of the SDSS field.]{Illustration of the filamentary network (black lines) for a slice of the SDSS field ($0.02 \leq z \leq 0.15$; $ 27 \leq$ dec $\leq 33$) extracted using the DisPerSE code. Only filamentary structures which are seen to persist above the 5$\sigma$ threshold are shown, along with the density contrast of the galaxy population. The density contrast is estimated using the small-scale DTFE estimator (see text). Adapted from \citet{duckworth2019_halo}, with credit to K. Kraljic.}
    \label{fig:disperse_sdss}
\end{figure*}

The local geometry of each filament is characterised by a series of smaller \textit{segments}, which are the default output of the DisPerSE code. For each galaxy, the nearest segment is found and its direction is used to compare to the given galaxy's spin direction. 

\subsubsection{Cosmic web distances} \label{sec:cosmic_web_distances}
Having constructed a skeleton of the cosmic web, a galaxy's environment can be described by finding its vicinity to various features of the skeleton. The cosmic web comprises of low density `void' regions which are enclosed by `walls' of structure which become filaments at points of intersection. The gravitational potential of the filaments dictate the flow of the matter, which at the point of intersection, feed high density regions interpreted as `nodes'. Along the filament, saddle points remain as minima between the flows towards nodes. 

The distance to the nearest filamentary point, $D_{skel}$, is first found for each galaxy. To then consider the influence of the nearest node, the distance from this impact point along the filament to the node is also computed, $D_{node}$. Finally the distance to the nearest wall, $D_{wall}$, can then be found. In order to investigate expected trends of galaxies with vicinity to any cosmic web feature we must remove effects resulting due to the proximity of others. For $D_{skel}$ we remove all galaxies that lie within $D_{node} < 0.5$ Mpc. 
% and for $D_{wall}$, all galaxies within $D_{node} < 0.5$ Mpc and $D_{skel} < 2.5$ Mpc are discounted. 
This represents a compromise between eliminating the effect of other cosmic web features and having enough galaxies left to construct a statistically significant sample. Tightening the condition with respect to nodes so that we require $D_{node} > 1$ Mpc does not change any of the results presented in this work. \red{find out maximum distance at which a galaxy is considered to be connected with a filament in Kat's sample.}

Construction of the cosmic web from any observation is influenced by the completeness and the sampling of the galaxy sample. The modified SDSS DR10 spectroscopic sample is complete to $m_r$ = 17.77. A sample containing only brighter galaxies will naturally only identify stronger/larger filamentary features and hence smaller substructures will be missed. In addition, the lower the sampling of galaxies, the lesser the accuracy of the actual position of cosmic web features. To correct for this, the distances are normalised by the mean inter-galaxy separation, $\left\langle D_z \right\rangle$ at a given redshift, as such $\left\langle D_z \right\rangle = n(z)^{-1/3}$ where $n(z)$ is the number density. \red{rewrite this paragraph}

\subsubsection{Angular momentum directions}
In order to estimate the spin direction of our galaxy sample, we assume a thin-disk approximation according to \citet{LeeErdogdu2007}. Here we summarise the key steps in calculating the three dimensional vector. Working in spherical coordinates in the reference frame of the galaxy, the spin direction can be described as
\begin{equation}
\begin{split}
\mathrm{\hat{L}_r & = \cos i,} \\
\mathrm{\hat{L}_{\theta} & = (1 - \cos^2 i)^{1/2} \sin PA,} \\
\mathrm{\hat{L}_{\phi} & = (1 - \cos^2 i)^{1/2} \cos PA,}
\end{split}
\end{equation}
where $\mathrm{PA}$ is the position angle from the stellar kinematics and $i$ is the inclination of the galaxy, defined as such
\begin{equation}
\mathrm{\cos^2 i = \frac{(b/a)^2 - p^2}{1 - p^2},}
\end{equation}
where $b/a$ is the sky projected axis ratio and $p$ is the intrinsic flatness of the galaxy \citep[varies as a function of morphology as described in][]{haynes1984}. $p$ accounts for the fact that the disk of galaxies has a finite thickness, and the presence of a bulge impacts the estimation of $b/a$. In this work we adopt an intermediate value (0.158), however choosing more extreme proposed values (0.1 - 0.23) does not change our findings. The value of $i$ is set to $\pi/2$ if $b/a < p$.

The spin direction can then be transformed into equatorial Cartesian coordinates as follows:
\begin{equation}
\begin{split}
    \hat{L}_x & = \hat{L}_r \sin \alpha \cos \beta + \hat{L}_{\theta} \cos \alpha \cos \beta - \hat{L}_{\phi} \sin \beta \\
    \hat{L}_y & = \hat{L}_r \sin \alpha \sin \beta + \hat{L}_{\theta} \cos \alpha \sin \beta + \hat{L}_{\phi} \cos \beta \\
    \hat{L}_z & = \hat{L}_r \cos \alpha - \hat{L}_{\theta} \sin \alpha
\end{split}    
\end{equation}
where $\alpha = \pi/2 - {\rm dec}$ and $\beta = {\rm RA}$.

% %We apply positive sign to $\hat{L}_r$ to all galaxies 
% %It is worth mentioning here that the spin vector determined from equation (9) suffers from the sign ambiguity in Lr, as mentioned in Pen et al. (2000) and Trujillo et al. (2006). Since it is not possible to determine the sign of Lr from the given information of the Tully Catalog, we apply positive sign to all Tully galaxies here. We expect that this sign ambiguity will play a role in decreasing the strength of the spin-shear alignment signal.
% \Kat{Talk about the ambiguity in the sign of $\hat{L}_r$ and the adopted positive value. Maybe we could do better, Chris I think you should have an information on whether it is positive of negative, no?}

\subsection{Spin alignment}
\subsubsection{LTGs}
\begin{figure}
    \centering
    \includegraphics[width=\linewidth]{thesis/latex/halo_assembly_manga/spin_fil_LTGs_2in1.pdf}
    \caption{Alignment between neighbouring filament and spin direction of late type galaxies calculated from the thin disk approximation. Each panel shows the probability density distribution (blue) of the cosine angle between the direction of the filament segment and the spin direction. Errors are calculated through bootstrapping and the grey line (with associated errors) corresponds to the expected distribution for a completely random alignment. In addition, each panel shows the mean $\cos \gamma$ for the population and the $p$-value for a Kolmogorov--Smirnov test. The left panel shows for all LTGs in our sample, whereas the right panel shows the distribution only for those with a stellar mass $< 10^{10} M_{\odot}$. We find that for all masses LTGs are preferentially \textit{parallel} orientated with respect to the neighbouring filament, with low mass showing a more significant alignment signal.}
    \label{fig:ltgs_spin_alignment} 
\end{figure}

In Figure \ref{fig:ltgs_spin_alignment}, we show the probability density function of alignment between the spin directions of LTGs and their neighbouring filament segment. We define $\gamma$ as the angle between the spin direction and segment directions. The PDF is presented as $|\cos \gamma|$ so that 0 corresponds to exact perpendicular alignment and 1 exact parallel alignment. Each panel shows the distribution for LTGs (blue) with associated bootstrap errors, along with the expectation of the distribution if the spin vectors have completely random orientations (grey line with errors). In addition, each panel shows the mean $\cos \gamma$ for the population and the $p$-value for a Kolmogorov--Smirnov (KS) test between the LTGs and the random orientations. A KS test evaluates if two sub-populations are drawn from the same distribution with the null hypothesis that they are consistent. A low $p$-value therefore signifies that the populations are independent to the significance level stated. 

In the left panel, we show the orientations for all LTGs selected in our sample. We find that the spin of LTGs are preferentially orientated \textit{parallel} to the direction of the neighbouring filament segment, to the significance level of $p_{KS} = 6.77 x 10^{-3}$. In the right panel, we show the PDF for only LTGs that are below a stellar mass of $\mathrm{M_{stel} < 10^{10} M_{\odot}}$. Again we find a preferential parallel alignment, however, now to an increased significance level of $p_{KS} = 3.02 x 10^{-6}$. In addition $\langle \cos \gamma \rangle$ increases from 0.53 (all LTGs) to 0.55 (low mass LTGs), indicative that this signal is strongest for the low mass LTGs. This parallel alignment is in-keeping with theoretical expectation that the formation of low mass disks tend to align with the large-scale tidal field in which they evolve. 

\subsubsection{S0s}
\begin{figure}
    \centering
    \includegraphics[width=\linewidth]{thesis/latex/halo_assembly_manga/spin_fil_S0s_2in1.pdf}
    \caption{Alignment between neighbouring filament and spin direction of lenticular galaxies calculated from the thin disk approximation. Each panel shows the probability density distribution (red) of the cosine angle between the direction of the filament segment and the spin direction. Errors are calculated through bootstrapping and the grey line (with associated errors) corresponds to the expected distribution for a completely random alignment. In addition, each panel shows the mean $\cos \gamma$ for the population and the $p$-value for a Kolmogorov--Smirnov test. The left panel shows for all S0s in our sample, whereas the right panel shows the distribution only for those with a stellar mass $< 10^{10} M_{stel}$. We find that for all masses lenticulars are preferentially \textit{perpendicular} orientated with respect to the neighbouring filament, with low mass showing a more significant alignment signal.}
    \label{fig:s0_spin_alignment}
\end{figure}

In Figure \ref{fig:s0_spin_alignment}, we now show the PDF of alignment between the disks of lenticular galaxies and their neighbouring filament segment. As before, we show the PDF as a $\cos \gamma$ distribution with the left panel showing the results for all S0s in our sample and the right for low mass ($\mathrm{M_{stel} < 10^{10} M_{\odot}}$) S0s only. Here, we find that the spin direction of S0s are preferentially \textit{perpendicular} with respect to their neighbouring filament segment. Considering the null hypothesis that the spin directions have random orientation, we find that the total S0 population is different from random to a significance of $p_{KS} = 8.56 x 10^{-3}$ which increases to  $p_{KS} = 8.37 x 10^{-4}$ when only considering the low mass S0s. This preferential perpendicular alignment could be indicative of the evolutionary history of lenticular galaxies embedded in large filamentary structure. The expectation from N-body simulations is that the orientation of dark matter haloes flip in direction due to mergers in the plane of the filament. This could be indicative that the S0s near filamentary structure have undergone mergers in their recent history, leading to a perpendicular orientation. 

\subsection{Discussion and Summary}
Of most relevance to our findings are the studies of spin alignment which also make use of large scale IFS surveys. In the SAMI survey, \citet{welker2020} correlate the spin directions of galaxies estimated from kinematics with the direction of the nearest filament segment (also defined using DisPerSE). They find evidence that low mass galaxies are preferentially parallel to filaments, whereas high mass galaxies are preferentially perpendicular (to a significance of 2$\sigma$). This is in agreement with expectations of a spin \textit{flip} seen for dark matter haloes in N-body simulations, due to initial preferential alignment (for low mass haloes) with the tidal field, before mergers in the plane of the filament cause a flip in orientation. Conversely in MaNGA \citet{krolewski2019} find no evidence for spin alignment with neighbouring cosmic web structure using the vector computed from stellar velocity fields. 

In both studies the spin alignment signal is computed using the spin and filament vectors in projected 2D space. While this reduces uncertainty when reconstructing the 3D spin vector (i.e. such as making an assumption of a thin disk approximation), it does not make use of the 3D information associated with filament reconstruction. Additionally making use of only the 2D information enables studies of galaxies without disks. Projecting the angle between two 3D vectors into 2D introduces possible projection degeneracies (see black histogram in Figure \ref{fig:PA_residual}) corresponding to a standard deviation of $\sim 13^{\circ}$ from the intrinsic value.

Our work represents the first study using IFS data to estimate 3D spin directions and correlating them with neighbouring filamentary structure. Our key findings are as follows: 
\begin{itemize}
    \item Spiral galaxies demonstrate preferentially \textit{parallel} spin orientations with respect to the nearest filament segment. The significance of the alignment signal is increased if we only consider low mass LTGs ($\mathrm{M_{stel} < 10^{10} M_{\odot}}$). 
    \item Lenticular galaxies demonstrate preferentially \textit{perpendicular} spin orientations with respect to the nearest filament segment. The significance of the alignment signal is increased if we only consider low mass S0s ($\mathrm{M_{stel} < 10^{10} M_{\odot}}$). 
\end{itemize}

Parallel spin alignment between LTGs and filaments is in agreement with \cite{welker2020} and expectations from N-body simulations \citep[e.g.][]{laigle2015}. Our finding of perpendicular alignment for lenticulars, especially for those low mass, is perhaps more surprising. This is in direct conflict with the expectation that low mass galaxies have a parallel orientation, which flips as they grow in mass. This could be indicative that the orientation is not only a function of mass, \textit{but also of morphology.} Following the idea that lenticulars form through mergers, this could be indicative of a spin \textit{flip} at lower masses as the galaxy becomes re-orientated which is reflected in the galaxy morphology. In the next section, we make further use of MaNGA to explore the connection between the cosmic web and galaxy kinematics, \textit{now} in the context of halo assembly.

\section{Kinematically misaligned galaxies and halo assembly} \label{sec:halo_assembly}
IFS observations provide kinematic information not only for the stellar continuum, but also for ionized gas ($\mathrm{H\alpha}$). As demonstrated throughout this thesis, we have investigated the relationship between kinematic misalignment, and fundamental properties such as galaxy \& halo spin, black hole growth \& feedback, and morphology. In this section we explore if misalignment can be used to trace dark matter halo assembly, and used as a proxy for the current rate of accretion onto central galaxies (and hence the current rate of DM accretion onto the halo). Following the discussion of \citet{hahn2009}, the current accretion rate is correlated with the large-scale tidal environment. Low-mass haloes in the vicinity of large haloes or massive structures will have their accretion `stalled' as tidal forces overcome their ability to accrete \citep[see also;][]{wang2007,dalal2008,lacerna2011}. Accordingly, this would lead to a relatively earlier formation time and a lack of on-going accretion seen today. Conversely low-mass haloes in environments of low magnitude tidal forces will continue accreting, and could have a late halo assembly time and continued gas accretion onto the central galaxy today. 

The relationship of environment and kinematic misalignment has previously been studied in MaNGA. Using a sample of 66 misaligned galaxies, \citet{jin2016} find that for all morphologies (or sSFRs), galaxies with misalignment between the rotation of their stars and gas are typically more isolated than their aligned counter-parts. In this section, we consider if these galaxies correspond to the later forming haloes. We explore whether position with respect to filamentary structures identified in the cosmic web, halo age and estimated group occupancy correlate with more recent accretion observed on the central galaxy of the group. 

\subsection{Introduction}
In this section we look for observational evidence of the effect of the cosmic web on the growth of haloes with the following tests:
\begin{enumerate}
\item Cosmic web environment (Section \ref{sec:cosmo_web_distances})
\begin{itemize}
\item Distances of haloes to filamentary structures are considered for galaxies split on $\Delta$PA. We test if low-mass haloes near filaments have their accretion `stalled' due to material preferentially flowing along the filament to more dense regions. 
\end{itemize}
\item Stellar to halo mass ratio  (Section \ref{sec:MsMh_hab})
\begin{itemize}
\item The stellar to halo mass ratio is used as a proxy for halo age and its correlation with $\Delta$PA is considered. We test if large-scale tidal forces can `stall' accretion onto low-mass haloes, seen as a low accretion rate today ($\Delta$PA < 30$^{\circ}$) on the central galaxy, indicating an earlier forming halo.
\end{itemize}
\item HOD modelling (Section \ref{sec:HOD_hab})
\begin{itemize}
\item A halo occupation distribution is constructed for groups with aligned and misaligned central galaxies. Earlier forming haloes provide more time for centrals to form and satellites to merge. This would correspond to a decrease in the magnitude of the HOD, which we aim to isolate. 
\end{itemize}
\end{enumerate}
In each sub-section we present both our method and results.

\subsection{Data}
\subsubsection{Stellar mass and halo mass definitions} \label{sec:mass_hab}
We use stellar masses estimated in the New-York University Value Added Catalogue from the K-correct routine \citep[NYU-VAC;][]{blanton2005}. To analyse the incident effects onto central galaxies as a function of environment, we require both group identification and estimations of the total halo mass. \citet{yang2007} (Y07 hence-forth) present an adaptive group finding algorithm based on the NYU-VAC to assign galaxies to haloes and then estimate and revise group properties through iteration. We will summarise the basic steps here. Potential group centres are first found through a friend-of-friends algorithm with small linking lengths in redshift space. All galaxies not currently linked are also considered as potential centres. For each tentative group, the combined luminosity of all group members with 
\begin{equation}
^{0.1}M_r - 5log(h) \leq -19.5
\end{equation}
are found, where $^{0.1}M_r$ is the absolute magnitude estimated from the NYU-VAC K-corrected to $z=0.1$. All galaxies within SDSS data-release 7 (DR7, all spectroscopic galaxies) below $z=0.09$ meet this criteria and hence the combined luminosity can be estimated directly. A corrective factor is applied for groups above this redshift. This total luminosity is then used to assign halo mass and various other group properties, which in turn refines the group identification following an iterative process until conversion. 

We remove all groups with $f_{edge} < 0.6$ as recommended by Y07 corresponding to groups on the survey borders. \citet{yang2009} provide a conservative estimate on the minimum halo mass for groups that would be expected to be complete as a function of redshift. This work primarily focusses on low-mass haloes, ($M_h \lesssim 10^{12.5} M_{\odot}$), however many of these groups will be incomplete at the redshift range selected. Y07 note however that any potential scatter arising from the incompleteness correction should be minimal in comparison with the scatter between group luminosity and assigned halo mass. 

The use of a group catalogue brings important considerations. In low-mass groups, the multiplicity of each group is low, and the luminosity (or stellar mass) of the central galaxy, will largely determine the mass of the group, potentially leading to an under-estimation of the scatter in the stellar-mass to halo-mass relation (see e.g. \citealt{campbell2015, reddick2013}). On the other hand, the fraction of groups with at least one satellite galaxy that is more massive than the central, increases steeply with halo mass (see e.g. \citealt{reddick2013}), leading to artificially larger scatter and an increased likelihood of central misclassification with increasing halo mass. \citet{campbell2015} demonstrate that group finder inferred measurements tend to equalise the properties of distinct galaxy sub-populations, however in general it is possible to recover meaningful physical correlations for average properties as a function of stellar and halo mass. 

We select only galaxies classified as centrals corresponding to the most massive galaxy within each group identified by Y07. The relationship between the central stellar mass and halo mass for galaxies within Y07 is shown in Figure \ref{fig:stel_halo_dist}. The increase of scatter in this relationship with increasing halo mass is likely explained by the limitations of group catalogues mentioned above. 

\begin{figure}
    \centering
	\includegraphics[width=0.8\linewidth]{thesis/latex/misalignment_MaNGA/stel_halo_ratio_dist.pdf}
    \caption[Hexagonal density plot of the relationship between the stellar mass  and the halo mass for central galaxies within Y07]{Hexagonal density plot of the relationship between the stellar mass  and the halo mass for central galaxies within Y07. The number counts are scaled logarithmically. The black lines show the median (solid) and the 20th and 80th quantiles (dashed) of the stellar mass in bins of halo mass.}
    \label{fig:stel_halo_dist}
\end{figure}

\subsubsection{Sample selection} \label{sec:samp_sec}
While the complete MaNGA sample may be unbiased to morphology, we must proceed with care when selecting a sub-sample usable for kinematic disturbance. To remove spurious PA fits we eyeball the entire MPL-6 sample and remove all galaxies which have a largely incomplete velocity field, poor or biased PA fit or are virtually face-on so that little or no rotational component is along the line of sight. This removes approximately half of MPL-6 observations. The majority of galaxies removed have largely incomplete gas velocity fields and for that reason our analysis naturally excludes gas-poor and slowly rotating elliptical galaxies. 

Recent studies have found that the fraction of slow rotators increases steeply with stellar mass and has a weak dependence on environment once this is controlled for \citep[e.g.][]{greene2018,lagos2018}. We note that our natural exclusion of gas-poor high-mass galaxies is reflected in the stellar mass distribution of the $\Delta$PA defined sample in Figure \ref{fig:samp_cons}. \red{Reference NSA target catalogue?}

\begin{figure}
    \centering
	\includegraphics[width=0.6\linewidth]{thesis/latex/halo_assembly_manga/sample_consistency.pdf}
    \caption[Relative frequency distributions of stellar mass and redshift for the NSA target catalogue, MaNGA MPL-6 and $\Delta$PA sub-samples.]{Relative frequency distributions of stellar mass and redshift for the NSA target catalogue (brown dashed line), MaNGA MPL-6 (green dot-dashed line) and our $\Delta$PA sub-sample (blue solid line). The figure is cut at $z=0.15$ representing the extent of MaNGA targets. Each histogram is given with Poisson errors on each bin.}
    \label{fig:samp_cons}
\end{figure}

GalaxyZoo1\footnote{At the point of submission for this work, GalaxyZoo1 provided the largest sample of galaxies with classified morphologies for the MaNGA survey. Since then a complete catalogue for all MaNGA targets has been constructed which was used and described in chapter 2 (see \S\ref{sec:morph_def_obs}).} provides visually identified morphologies for a large sample of SDSS galaxies \citep{lintott2008}. Morphology is identified by having over 80\% of debiased classification votes in the same category (i.e. elliptical or spiral). The remainder of galaxies are marked as uncertain morphology. We compare the fraction of ellipticals in our $\Delta$PA sub-sample with MPL-6, as shown in Table \ref{tab:GZ}. GalaxyZoo1 only provides classifications for 3/4 of MPL-6 as reflected by the total classification numbers. We find the fraction of ETGs falls from 0.242 to 0.111, reaffirming our bias against slow-rotating high-mass ellipticals that our eye-balling tends to remove. 

If galaxies are truly morphologically transformed, then this should be reflected in their angular momentum. \citet{cortese2016} find that galaxies lie on a tight plane defined by stellar angular momentum ($j_{stars}$), S\'ersic index and stellar mass when excluding slow rotators in the SAMI galaxy Survey. This could indicate that fast rotating early-type and late-type galaxies are not two distinct populations but instead represent a continuum connecting pure-discs to bulge dominated systems \citep{cappellari2011}. This can be linked to simulation: \citet{lagos2017} use EAGLE \citep{EAGLE2015} to investigate the effects of galaxy mergers on the evolution of stellar specific angular momentum. They find that the gas content of a merger is the most important factor for dictating $j_{stars}$ for the remnant, ahead of both the mass ratio and spin/orbital orientation of the merger progenitors. An increasing rate of wet (gas-rich) mergers corresponds to decreasing stellar mass and increasing $j_{stars}$. Conversely dry (gas-poor) mergers are the most effective way of spinning down galaxies, with gas-poor counter-rotating progenitors creating the biggest decrease in $j_{stars}$. Following this narrative, it is fair to exclude slow rotators which could follow a different evolutionary track to a continuum of fast rotating galaxies in angular momentum phase space. 

We note that the initial GalaxyZoo1 classifications mirror our findings in Chapter 2 with a larger sample. Despite the $\Delta$PA defined galaxies in this analysis being predominately classified as LTGs, we find the majority of kinematically misaligned galaxies are ETGs. While understanding how a morphology bias could impact any results presented, we direct the reader to Chapter 2 for a more complete discussion on the relationship between misalignment and morphology.

\begin{table}
\centering
\begin{tabular}{|l|c|c|c|c|}
\hline
& Total & GZ1 & ETG & LTG \\ \hline
MPL-6 & 4614 & 3598 & 869 (0.242) & 1225 (0.340) \\
$\Delta$PA defined & 2272 & 1835 & 204 (0.111) & 1005 (0.548) \\
$\Delta$PA > 30$^{\circ}$ & 192 & 151 & 85 (0.556) & 9 (0.060)\\
Final sample & 925 & 812 & 136 (0.167) & 456 (0.561) \\ \hline
\end{tabular}
\caption{(Rows: top to bottom) All usable MPL-6 galaxies, all $\Delta$PA defined galaxies within MPL-6, those that are kinematically misaligned with $\Delta$PA > 30$^{\circ}$ and the final sample of central, $\Delta$PA defined galaxies used in this Chapter. For each row, the total number of galaxies is given, along with those defined in GalaxyZoo1 (denoted GZ1 in table) and the total number of which that are classified into early-type (ETG) and late-type (LTG). The fractions of early-types and late-types are defined with respect to the total number of GalaxyZoo1 defined galaxies.}
\label{tab:GZ}
\end{table}

We are looking for accretion due to large-scale influence, so we remove all obvious on-going mergers through visual inspection of both the field photometry and IFU observations. We also identify target galaxies interacting with close pairs or neighbours. While this visual inspection should identify the majority of on-going major mergers, we note that our identifications are clearly subjective. We remove $\sim$50 galaxies, identified to be merging or interacting with a nearby neighbour. After matching to the Y07 group catalogue for halo mass we are left with 925 central galaxies which we use in this chapter. 

\subsection{Cosmic web distances and local environment} \label{sec:cosmo_web_distances}
\subsubsection{Over-density and stellar mass}
Before considering cosmic web environment, we first consider the role of environmental density and stellar mass in our results. A dependence on small scale density is an indirect effect of halo mass and would not probe the large-scale anisotropy of the cosmic web. It is also important to isolate the role of stellar mass and morphology following their distinct gradients with respect to cosmic web features found in GAMA \citep{kraljic2018}.

\begin{figure*}
    \centering
	\includegraphics[width=\linewidth]{thesis/latex/halo_assembly_manga/PA_ALL_ET_DENSITY.pdf}
    \caption[Probability density distributions of density smoothed with a Gaussian kernel at the scale of 3 Mpc \& 9 Mpc and the stellar mass, log$_{10}(M_{\ast}/M_{\odot})$ for all $\Delta$PA defined galaxies and all GalaxyZoo1 classified elliptical galaxies with a $\Delta$PA.]{Probability density distributions of density smoothed with a Gaussian kernel at the scale of 3 Mpc \& 9 Mpc and the stellar mass, log$_{10}(M_{\ast}/M_{\odot})$ (left to right) for all $\Delta$PA defined galaxies (top row) and all GalaxyZoo1 classified elliptical galaxies with a $\Delta$PA (bottom row). Aligned galaxies ($\Delta$PA < 30$^{\circ}$) are shown in red (solid line) and those with high misalignment ($\Delta$PA > 30$^{\circ}$) are in black (dashed line). Each histogram is given with Poisson errors on each bin. A two-sample KS statistic and its corresponding p-value is overlaid for comparison between the distributions in each cell and the vertical lines denote the corresponding distribution's median. Using all galaxies, the misaligned sample resides in higher density at the scales of 3 Mpc and 9 Mpc to the aligned sample respectively, but are equivalent in stellar mass. Selecting only ETGs accounts for the difference in small scale density but the misaligned sample are at lower stellar masses than the aligned.}
    \label{fig:density_hab}
\end{figure*}

Figure \ref{fig:density_hab} shows the distributions of densities and stellar mass for the aligned and misaligned samples. Density used in this work is computed using a Delaunay tessellation of the discrete galaxy positions through the DTFE estimator, smoothed with a Gaussian kernel at local scales (3 Mpc) and at large scales (9 Mpc). 

We evaluate the likelihood of the aligned and misaligned sample being drawn from the same continuous distribution through implementation of a two-sample Kolmogorov--Smirnov (KS) test. In each case a D-value of the KS statistic with a corresponding p-value is provided. The D-value (referred to as the KS statistic hence-forth) provides the maximum fractional difference between the cumulative distribution functions with a p-value corresponding to the null hypothesis that the two samples are drawn from the same continuous distribution. A high KS value combined with a low p-value (for example; KS $\geq$ 0.1 for P $\leq$ 10\% confidence level) therefore is consistent with the two samples being significantly different.

The first row of Figure \ref{fig:density_hab} considers the difference between all $\Delta$PA defined galaxies. We find that the misaligned galaxies ($\Delta$PA > 30$^{\circ}$) reside in more dense environments at small and large scales with a probability that the distributions are instead consistent of 0.3\% and 0.5\% respectively. The two samples are consistent in distribution of stellar mass, despite the difference in classified morphology, as shown in Table \ref{tab:GZ}. In the second row of Figure \ref{fig:density_hab} we consider the same properties but only for ETGs. We select on optical morphology using the classifications of GalaxyZoo1 as introduced in \S\ref{sec:samp_sec}. We find that the difference seen in density smoothed on small scales may be explained by morphology as the p-value for the KS test increases, however misaligned galaxies tend to reside in more dense large-scale environments and populate lower stellar masses compared with aligned galaxies of the same morphology.

In order to minimize the effect of $\rho_{3Mpc}$ and $M_{\ast}$ in our cosmic web results, in the next section we weight distance distributions on both stellar mass and small scale density, when comparing the distribution of cosmic distances for the aligned and misaligned samples. This is done through normalising the histogram of the weight quantity to be consistent between distributions using a minimum of three bins. We also include the results of ETGs only to minimize the impact of morphology. We cannot do the same for LTGs due to the lack of misaligned LTGs. 

\subsubsection{Results of cosmic web distances} \label{sec:cw_res}

\begin{figure}
    \centering
	\includegraphics[width=0.85\linewidth]{thesis/latex/halo_assembly_manga/PA_ALL_CW.pdf}
    \caption[Probability density distributions of normalised distances to cosmic web features for all $\Delta$PA defined galaxies.]{Probability density distributions of normalised distances to cosmic web features for all $\Delta$PA defined galaxies. Distances to nodes (left) and filaments (right) are normalised by the sampling at a given redshift. The distributions of galaxies in the top row is weighted on stellar mass between the aligned (red solid line) and misaligned samples (black dashed line). The distributions are weighted by density smoothed by a Gaussian kernel at the scale of 3 Mpc for the middle row and are left unweighted for the bottom row. A two-sample KS statistic and its corresponding p-value is overlaid for comparison between the distributions in each cell. The error bars represent the poisson noise in each bin. The weighted median values for each distribution are shown by the vertical lines.}
    \label{fig:cw_all}
\end{figure}

Figure \ref{fig:cw_all} shows the distance probability density distributions of aligned and misaligned galaxies with respect to nodes (left) and filaments (right). The top row shows the two samples weighted on stellar mass, the middle row weighted on small scale density (3 Mpc smoothed) and the bottom row shows the raw distributions. The results of a two-sample KS test with corresponding weightings to the cumulative distribution function are overlaid in each cell. 

The distributions of aligned and misaligned galaxies with respect to filaments meet the null hypothesis criterion of high p-values for all weighting schemes (i.e. no statistically significant difference between distributions). This is indicative that $\Delta$PA is independent of the influence of filaments identified in our analysis. For the unweighted samples, we find that misaligned galaxies typically reside in closer vicinity to nodes than their aligned counterparts as indicated by a p-value of 0.089. This difference is however partially negated by weighting on stellar mass or density smoothed on the 3 Mpc scale, as reflected in slightly reduced KS values and p-values increased above the 0.1 significance level.

\begin{figure}
    \centering
	\includegraphics[width=0.9\linewidth]{thesis/latex/halo_assembly_manga/PA_ET_CW.pdf}
    \caption{Same as Figure \ref{fig:cw_all} but using only visually selected ETGs as found in GalaxyZoo1.}
    \label{fig:cw_et}
\end{figure}

The origin of misaligned galaxies residing preferentially closer to nodes could be explained by their morphology difference with respect to the aligned sample. In previous work, \citet{kraljic2018} found distinct gradients of stellar mass and morphology with vicinity to nodes and filaments. Figure \ref{fig:cw_et} shows the distributions of cosmic web feature distances but now only selecting ETGs. We find that in all weighting schemes, the distance distributions of aligned and misaligned galaxies with respect to both nodes and filaments meet the null hypothesis criterion as reflected in large p-values (> 0.4). These distributions appear to be drawn from the same continuous distribution, indicating that direct and indirect effects of morphology are likely responsible for the difference in distance to nodes. 

\subsubsection{The role of halo mass} 
Our primary aim in this section is to isolate if vicinity to filaments can impact the rate of present-time accretion on a central galaxy in a low-mass halo. Including high-mass haloes in our sample may counteract any observable signal as they are possible candidates responsible for quenching accretion. High-mass, typically old haloes, are the opposite of what we are trying to target: young, still forming low-mass haloes (with respect to old low-mass haloes). 

We now consider $D_{skel}$ for low-mass haloes only. \citet{tojeiro2017} found signal of halo assembly bias in low-mass haloes using the stellar to halo mass ratio in GAMA. Low-mass haloes residing in regions of stronger tidal forces were found to form earlier irrespective of density, with this trend apparently reversed at high mass. This signal was found to be strongest for haloes of mass $M_h \sim 10^{12.3} M_{\odot}$, however a slight trend was found even at $M_h \sim 10^{12.74} M_{\odot}$. To ensure we have enough objects for a statistically significant sample we therefore consider all central galaxies residing in haloes of mass: $M_h \sim 10^{12.5} M_{\odot}$. Figure \ref{fig:mh_cw} shows the distance probability density distributions for the whole $\Delta$PA defined sample and GalaxyZoo1 defined ETGs only. For all weighting schemes, we find p-values consistently above the 0.1 significance level and hence conclude the null hypothesis that the aligned and misaligned galaxies are consistent in distance distributions with respect to filaments. This holds true regardless of morphology selection.

\begin{figure}
	\includegraphics[width=\linewidth]{thesis/latex/halo_assembly_manga/PA_ALL_CW_Mh12_5.pdf}
    \caption[Probability density distributions of normalised distances to filaments for galaxies in MaNGA MPL-6.]{Probability density distributions of normalised distances to filaments for galaxies with log$_{10}$(M$_h$/M$_{\odot}) \le 12.5$. As in Figure \ref{fig:cw_all} the distributions are weighted on stellar mass (top), density smoothed at scales of 3 Mpc (middle) and left unweighted (bottom). The distributions of all $\Delta$PA defined galaxies (left) and only galaxyZoo selected ETGs (right) are shown for comparison.}
    \label{fig:mh_cw}
\end{figure}

\subsection{Stellar to halo mass ratio}\label{sec:MsMh_hab}
In this section we introduce an observational proxy for halo formation time: the central stellar to total halo mass ratio. Its use was motivated in \citet{wang2011} who explored the correlation between various halo properties which were identified within a set of seven N-body simulations using the P\textsuperscript{3}M code described in \citet{jing2007}. One of the most important properties is its formation time, $z_f$ which was shown to correlate with galaxy properties such as SFR, galaxy age and colour. The formation time in this instance is defined as the redshift at which the main progenitor has formed half of the mass of the final halo. %An observational counterpart must be constructed as $z_f$ cannot be determined from galaxies alone. 
They establish that formation time shows a tight correlation with the sub-structure fraction, $f_s = 1 - M_{main}/M_h$, where $M_{main}$ and $M_h$ are the main \textit{sub}-halo mass and the halo mass respectively \citep{gao2007}. 
The ratio of $M_{main}$ and the total halo mass is seen to act as an robust estimate for formation time. Following \citet{lim2015}, we use the following as an observational proxy,
\begin{equation}
f_c = \frac{M_{*,c}}{M_h},
\end{equation}
where $M_{*,c}$ is the stellar mass of the central galaxy. The $M_h$ in this instance is found using the group stellar mass ranking from Y07. $M_{*,c}$ is a reasonable estimator for the main sub-halo mass $M_{main}$ however they do not hold an exactly monotonic relation. Given this and that $f_s$ is not perfectly correlated with formation time, $f_c$ can only be considered to be a relative proxy of $z_f$ as shown in \citet{lim2015}. A higher value of $M_{*,c}/M_h$ should correspond to a relatively older halo. 
Semi-analytic and hydrodynamical simulations have since confirmed a correlation between halo assembly time and the stellar to halo mass ratio, and have shown how halo formation time partly explains the scatter in the stellar mass to halo mass relation \citep[e.g.][]{matthee2017,tojeiro2017,zehavi2018}. Observationally, \cite{tojeiro2017} show that the stellar to halo mass ratio of central galaxies varied with position within the cosmic web, at fixed halo mass. In this section, we investigate whether recent accretion history, associated with younger halos, might be visible in the kinematics of gas and stars.

\subsubsection{Results of the halo age proxy}
Figure \ref{fig:2d_ratio} shows the stellar to halo mass ratio as function of $\Delta$PA. Since we do not possess errors for stellar mass or halo mass and can only roughly estimate $\Delta$PA errors, we bin our data and calculate the standard error on the mean. We split our sample at the median halo mass of $M_{\ast} = 10^{12.3} M_{\odot}$ and divide galaxies into bins with boundaries; $\Delta$PA$ = [0,5,10,20,90,180] (^{\circ})$. In each of the halo mass bins, we weigh the redshift and halo mass distributions to be consistent in each bin of $\Delta$PA. Figure \ref{fig:2d_ratio} shows no particular dependence on $\Delta$PA for this proxy of halo age. However, given the strong dependence of $M_{\ast}/M_{h}$ on $M_{h}$, we investigate this further by simultaneously considering the relationship between $M_{h}$, $M_{\ast}/M_{h}$ and $\Delta$PA. 

\begin{figure}
    \centering
	\includegraphics[width=0.8\linewidth]{thesis/latex/halo_assembly_manga/stel_halo_ratio_bin0_10_20_30_90.pdf}
    \caption[The central stellar to total halo mass ratio for the $\Delta$PA sub-sample.]{The central stellar to total halo mass ratio for the $\Delta$PA sub-sample in bins of halo mass. Galaxies residing in haloes of M$_{\ast} < 10^{12.3}$M$_{\odot}$ (blue dashed line), M$_{\ast} > 10^{12.3}$M$_{\odot}$ (red dot-dashed line) and the total sample (black solid line) are divided into bins of $\Delta$PA$ = [0,10,20,30,90,180] (^{\circ})$. Error-bars are given by the standard error on the mean. Within each bin of halo mass, distributions are weighted on redshift and halo mass between all $\Delta$PA bins.}
    \label{fig:2d_ratio}
\end{figure}

%Figure \ref{fig:plane_fit} shows the three parameter space. 
We divide our parameter space into quarters by splitting galaxies at $\Delta$PA = 30$^{\circ}$ and $M_{h} = 10^{12.3}M_{\odot}$. In each region we fit a flat plane to our data points as described by,
\begin{equation}
M_{\ast}/M_{h} = c_{0} log_{10}(M_{h}/M_{\odot}) + c_{1}\Delta PA + c_{2}.
\end{equation}
A strong correlation between $M_{\ast}/M_{h}$ and $\Delta$PA would correspond to a relatively large value of $c_{1}$ with regards to $c_{0}$. To understand the significance of any result, we also fit a flat plane with $c_{1} = 0$ (i.e. no dependence on $\Delta$PA) and evaluate $\chi_{red}^2$ for both. These values are found in Table \ref{tab:chisq}. We are inherently limited by having no errors on the estimates of stellar mass and halo mass for our sample. We therefore construct constant errors across the sample estimated from the sample variance. The sample variance itself is found from considering each data point with regards to its 10 nearest neighbours in the parameter space. 

We find that fixing the gradient along $\Delta$PA has little or no effect on the fit of the linear plane, regardless of how we sub-divide our parameter space. In some cases the comparison planes are effectively the same allowing for a smaller $\chi_{red}^2$ for the two free parameter fit. As discussed in \S\ref{sec:mass_hab}, halo masses assigned to galaxy groups using the Y07 group catalogue are corrected due to incompleteness above $z=0.09$. We consider the plane fitting again but with redshift cuts at both $z=0.09$ and a conservative $z=0.05$. In both instances, we also find there are no statistically significant gradients along $\Delta$PA. We therefore conclude that $\Delta$PA holds little correlation with the age of the halo in which it resides, as inferred from current measurements of $M_{\ast}/M_{h}$. 

% \begin{figure}
% 	\includegraphics[width=\linewidth]{stel_halo_param_chisq_fit}
%     \caption{Parameter space for the central stellar to halo mass ratio (M$_{\ast}$/M$_{h}$), halo mass (M$_{h}$) and $\Delta$PA. $\Delta$PA and M$_{h}$ are scaled logarithmically for presentation purposes. All scatter points correspond to galaxies within our analysis and are fit by 2D planes through least squares minimization. Each colour corresponds to the parameter space range in which a plane was fitted. The parameter space is divided at $\Delta$PA = 30$^{\circ}$ corresponding to red and black filled colours below and above this. It is secondly split at $M_{h} = 10^{12.3}M_{\odot}$ with blue and red edge colours corresponding to low and high mass respectively.}
%     \label{fig:plane_fit}
% \end{figure}

\begin{table}
\centering
\begin{tabular}{|l|c|c|c|}
\hline
$M_{h}/M_{\odot}$& & $\Delta$PA = [0, 30]($^{\circ}$) & $\Delta$PA = [30, 180]($^{\circ}$) \\ \hline 
[$10^{11.7}, 10^{12.3}$] & $c_{1} = 0$: & 1.188 & 1.070 \\
					   & Free : & 1.190 & 1.068 \\ \hline
[$10^{12.3}, 10^{14}$]   & $c_{1} = 0$: & 1.218 & 0.988 \\ 
					   & Free: & 1.212 & 1.001 \\ \hline
% fix units on this table
\end{tabular}
\caption{$\chi_{red}^2$ for plane fits with $c_1 = 0$ and left free in the parameter space for the central stellar to halo mass ratio (M$_{\ast}$/M$_{h}$), halo mass (M$_{h}$) and $\Delta$PA. The parameter space is divided at $\Delta$PA = 30$^{\circ}$ and $M_{h} = 10^{12.3}M_{\odot}$.}
\label{tab:chisq}
\end{table}

\subsection{Halo occupation distribution} \label{sec:HOD_hab}
In this section we introduce the HOD function and how it can be used to infer halo age. In describing the relationship between galaxies and dark matter haloes, HODs are a useful prescription to determine models of galaxy formation and evolution \citep[e.g.][]{berlind2003}. They provide a probability distribution function $P(N|M_h)$ for a set of virialised haloes where $N$ is the number of hosted galaxies for a given halo mass $M_h$. A fundamental assumption underlying HOD modelling is that the galaxy occupation is purely dependent on the halo mass. Typically the observed galaxy clustering is used to construct the empirical relationship that allows mock dark matter haloes to be populated with galaxies. Assembly bias would directly affect the observed clustering of galaxies and hence challenge any interpretation using the HOD framework. 

Continuing our discussion, low-mass haloes near large haloes are expected to cease formation earlier. This leads to a boost of galaxy clustering at this halo mass range relative to the overall sample as they live preferentially in high density regions. \citet{zehavi2018} previously investigated the dependence of occupation functions on various properties such as large-scale environmental density and halo age using semi-analytical galaxy models applied to the Millennium simulation \citep{springel2005} \citep[See also;][who confirmed these results using the hydro simulations of EAGLE and Illustris]{artale2018}. They find that higher density environments generally act to populate lower mass haloes with central galaxies. A stronger dependence can be found on halo age, however, as earlier forming low-mass haloes are more likely to host central galaxies. In addition, earlier forming haloes are likely to host fewer satellites relative to late forming haloes at fixed halo mass. A simple explanation is that the early forming haloes provide more time for their constituent satellite galaxies to merge with the central. More massive central galaxies may therefore reside in low-mass haloes that formed early due to this general in-flow, analogous to a higher stellar to halo mass ratio. 

\subsubsection{Background subtraction}
To understand the assembly history of a central galaxy's sub-halo we must consider the role of satellites that contribute to the hierarchical structure growth of its halo merger tree. However, we are limited by the magnitude and typical size of galaxies inhibiting small substructure around a main sub-halo. \citet{liu2011} demonstrate a common method for counter-acting the lack of spectroscopic information for satellite galaxies through counting possible photometric group members. Their numbers are then statistically corrected to remove the contribution of contaminant foreground and background galaxies outside of the group. This enables a lower limit of apparent magnitudes which can be accessed through use of the background subtraction technique. \citet{rodriguez2015} extend this formalism to HOD modelling and provide the technique we implement here. For a complete description of the technique we direct the reader to this reference, however we will summarise the basic concepts here.

Background subtraction requires two catalogues that share the same sky area; we will use our $\Delta$PA defined MaNGA centrals with their identified groups in combination with the photometric SDSS catalogue. For the photometric galaxies in the sky region of a group, their absolute magnitudes are calculated at the redshift of the group, $z_{f}$. The total number of galaxies with an absolute magnitude $M \le M_{min}$ are then counted within a circle around the group centre with its radius determined by the projected characteristic radius on the sky. In order to remove background galaxies, an estimation of the local density with respect to the average catalogue density must be made. All galaxies with $M \le M_{min}$ are recounted in concentric annuli centred on the group to provide the local density. A correction for the total number of galaxies in the group can then be estimated by subtracting the local background density multiplied by the group's projected area. The HOD is then constructed by binning the groups into mass intervals and averaging.

\citet{rodriguez2015} demonstrate the recovery of the background subtraction method using mock catalogues constructed from semi-analytic models of galaxy formation applied on top of the Millennium simulation. They compare the HODs found from the background subtraction technique to HODs of the direct galaxy counts in volume limited samples for different magnitude limits \citep[see Figure 1 in;][]{rodriguez2015}. Beyond a small overprediction for fainter magnitudes ($M_{lim} \approx -16.0$), they find good agreement with direct galaxy counts for all absolute magnitudes.

Results from the background subtraction technique have also been compared to results of other HOD estimation techniques applied to observations. \citet{yang2008} parametrise HODs for satellite galaxies in groups identified in SDSS DR4 using the adaptive group-finding algorithm of Y07 (see \S\ref{sec:mass_hab} for discussion). The background subtraction technique shows great agreement in estimating parameters of the HOD relative to the method of \citet{yang2008} but additionally offers the ability to estimate the HOD for fainter absolute magnitudes than previous work \citep[see Figure 5 in][]{rodriguez2015}.

MPL-6 does not provide a large enough sample size to construct a reliable two-halo term in HOD modelling through calculation of the cross-correlation function. Background subtraction therefore represents the best estimation for this sample size.

\subsubsection{Results of the HOD}
We match each central galaxy in our $\Delta$PA sample with its corresponding satellite group members using \citet{yang2007}. We split our groups at $\Delta$PA = 10$^{\circ}$ for the central and calculate the HOD using the background subtraction technique. Our lower split in $\Delta$PA is purely due to limitations of sample numbers. As demonstrated in \S \ref{sec:kin_mis}, this should be above the resolution limit of $\Delta$PA, however may include more galaxies with spurious kinematic misalignments or due to an internal origin. 

Figure \ref{fig:hod_mpl6} shows the HOD for different magnitude cuts of the group during comparison to the photometric background. As fainter galaxies are removed, the overall magnitude of the HOD naturally decreases as we have less complete groups. As a sanity check, we compare our HODs estimated from the background subtraction technique to HODs estimated directly from clustering in SDSS DR7 with similar magnitude cuts \citep{zehavi2011}. The authors use measurements of the projected correlation function for SDSS DR7, which is translated into a HOD through use of a smoothed step function \citep[see equation 7;][]{zehavi2011}. This comparison is shown in panels II-IV (green dashed line) and matches our estimation well.
Regardless of $\Delta$PA classification, at all magnitude thresholds we find that the difference between the HODs are indiscernible. We conclude that $\Delta$PA of the central galaxy does not produce a difference in its constituent group assembly that can be seen through occupation functions. A detailed analysis using cross-correlations will be presented in future work, once the sample size of MaNGA is sufficient. 

\begin{figure}
	\includegraphics[width=\linewidth]{thesis/latex/halo_assembly_manga/hod_vs_zehavi.pdf}
    \caption[Halo occupation distributions using background subtraction for groups with central galaxies split on $\Delta$PA.]{Halo occupation distributions using background subtraction for groups with central galaxies with $\Delta$PA < 10$^{\circ}$ (red) and $\Delta$PA > 10$^{\circ}$ (black). Panels I, II, III \& IV correspond to a $r$-band magnitude cut of $\leq$ -17, -18, -19 and -20 respectively. The points reflect the estimation for individual groups, with these lines representing the mean with corresponding errors on the mean. In panels II-IV, comparison to HODs estimated directly from clustering with similar magnitude cuts are shown by the green dashed lines \citep[][see text]{zehavi2011}.}
    \label{fig:hod_mpl6}
\end{figure}

\subsection{Discussion} \label{sec:halo_assembly_discussion}
An important assumption in our motivation has been that gas accretion should originate from cold filamentary flows of the cosmic web. In reality, gas in-flowing into a central galaxy could also result from cooling of the surrounding hot halo and accreting in a more stochastic nature, so we consider that next. 

As introduced in \S \ref{sec:kin_mis}, \citet{lagos2015} explore the origin of kinematic misalignment between gas and stars in ETGs using GALFORM, in comparison with the misaligned field ETG fraction found in ATLAS\textsuperscript{3D} of 42 $\pm$ 6\%  \citep{davis2011a}. They find that using solely galaxy mergers as the source for misaligned cold gas only predicts 2\% of ETGs to have $\Delta$PA > 30$^{\circ}$. Regardless of the time-scales of dynamical friction used, there are simply not enough mergers at $z=0$ to recreate the misaligned fraction observed. To include the effects of smooth gas accretion, they trace its history onto a subhalo along with incident galaxy mergers in their GALFORM model. They follow the angular momentum flips in the constituent cold gas, stars in galaxies and the corresponding dark matter halo using the Monte Carlo simulation prescription of \citet{padilla2014}. This simulation analyses the incident mass with respect to the subhalo, categorising by source (smooth accretion or merger) and constructs a PDF of the expected change in rotational direction for each component: hot halo, cold gas disc and stellar disc. \citet{padilla2014,lagos2015} consider the gas and stellar discs of galaxies to be initially aligned with the surrounding hot halo of gas from which they cooled. When a dark matter halo is accreted, the hot halo is immediately offset from the original rotation, which in time cools to create a misaligned gas disc in the galaxy. Memory of the misalignment can be erased through disc instabilities which use cold gas in the form of a starburst. It should however be noted that this model does not include the relaxation of the gas disc towards the stellar component due to torques. With this in mind any expected misalignment can only be considered an upper limit. \citet{lagos2015} reproduce consistent fractions of misalignment with ATLAS\textsuperscript{3D} by assuming that accretion does not come from a correlated preferential direction. To consider the effect of filamentary `cold mode' accretion on misalignment, the direction of accretion is then correlated on various time-scales and again the expected misaligned fraction is calculated. Assuming an uncorrelated direction of accretion marginally better reproduces observations but more importantly highlights the important role of slower `hot mode' accretion in interpretation. Stochastic accretion onto the galaxy from the hot halo may be the driving factor in misalignment of the gas disc, explaining the lack of correlation with our measures of large-scale environment and halo age. 

\citet{correa2018} investigated the role of cold and hot modes of accretion onto galaxies with respect to the accretion rate onto the host DM halo using the EAGLE suite of hydrodynamical cosmological simulations. In haloes of mass > $\mathrm{10^{12}M_{\odot}}$, the two modes of accretion coexist and both contribute to the gas accretion rate on central galaxies. Below this value the cold mode of filamentary flows appear to dominate whereas the hot mode dominates above $\mathrm{10^{12.7}M_{\odot}}$ for $z = 0$.  They note that AGN feedback plays an important role on the ability of gas from the surrounding hot halo to cool and accrete and is likely less efficient at high halo masses explaining why hot mode accretion becomes dominant. The ability of cold flows to reach the halo centre is, however, unconfirmed. \citet{nelson2013} find that the majority of gas from cold mode accretion is shock heated as it travels from the DM halo. They compare the differences of the moving mesh code AREPO with the results of GADGET-3 using otherwise identical simulation runs. While gas filaments in GADGET remain collimated and flow coherently to small radii, the same filamentary gas streams in AREPO are heated and become disrupted around  0.25-0.5 r$_{vir}$, boosting the rate of hot gas accretion as a result. The prominence of cold and hot modes of accretion and their subsequent ability to misalign the gas component of a galaxy is rightfully under debate. The lack of correlation of kinematically misaligned galaxies with environments of expected continued accretion could simply indicate that hot mode accretion is dominant in these regimes. 

Another consideration is the visibility of accretion within the effective radii observed in MaNGA. The Primary+ galaxy sample (63\% of MaNGA total including both Primary and Colour Enhanced samples) observes galaxies up to a minimum of 1.5\re, whereas the Secondary sample (37\% of MaNGA total) goes to a minimum of 2.5\re. All modes of accretion would be expected to be most visible on the outskirts of the central galaxy which could be further than 1.5-2.5\re. Below this we could expect gas and stellar components to align on much faster time-scales after an accretion event due to the strength of stellar torques peaking closer to the galactic centre. Despite this, it should be noted that approximately only 20\% of galaxies with $\Delta$PA > 30$^{\circ}$ are from the Secondary sample whereas the fraction of aligned galaxies in the Secondary is as expected from the targeting. 

To assess the impact of changing the observation extent between 1.5 and 2.5\re on $\Delta$PA, we consider all Secondary sample galaxies. We find no significant difference in $\Delta$PA when fitting to an aperture 0.6 (1.5/2.5) of the total size of the original IFU extent. This could be a natural limitation of $\Delta$PA being an average property over all radii and hence being preferentially biased towards the rotation of its likely kinematically aligned centre when considering a population excluding recent mergers. The probability of misalignment is linked to the mass of accreted material and lower mass accretion may well propagate to `warps' in the gas velocity map (i.e. $\Delta$PA changing as a function of radius) while maintaining an aligned classification. During visual inspection we found the scenario of a warped gas map while maintaining an undisturbed stellar velocity field to be rare (seen in approximately 20 galaxies). Bars could also be attributed to create warps in velocity fields. \red{Update this statement.} We look to \citet{stark2018} who implement a modified radon transform to characterise PA and its radial variation in the velocity fields of MaNGA for the prevalence of these effects.

Finally, we consider the impact of using a group catalogue to identify central galaxies and estimate halo masses. As discussed in \S\ref{sec:mass_hab}, halo mass is less accurate for small groups, and central mis-classification is more problematic at large halo mass. An estimate of halo mass is only important in one of our tests, where we consider the stellar to halo mass ratio as a proxy for halo formation time. It is possible that errors in halo mass estimates simply averaged out any real signal of $\Delta$PA with the stellar to halo mass ratio. Whereas our two other tests use halo mass estimates to split the data into two populations, the dependence on halo mass values is much reduced, and the Y07 catalogue has been shown to reproduce general trends of galaxy properties as a function of halo mass well \citep{campbell2015}. 
Mis-classification of central galaxies has implications throughout our paper. However at $\mathrm{M_h < 10^{13} M_{\odot}}$, where effects of halo assembly are expected to be more prominent, the fraction of groups where stellar ranking results in mis-identification of a satellite as a central is estimated to be well within 10\% by \cite{campbell2015} and \cite{reddick2013}. To consider how 10\% mis-classifications could impact Figure \ref{fig:2d_ratio}, we perform 50 realisations where we remove 10\% of our central sample and replace these with satellites (with their own defined $\Delta$PA) with a consistent distribution in halo mass. This shown in Figure \ref{fig:2d_ratio_realisations} where the additional realisations are plotted in the same colour with different transparency. The overall amplitude of $\mathrm{M_{\ast}/M_{h}}$ tends to decrease (especially for low halo mass groups), however there appears to be no noticeable changes in trend. This is expected if a signal is not strong with either population of galaxies.

\begin{figure}
    \centering
	\includegraphics[width=0.7\linewidth]{thesis/latex/halo_assembly_manga/halo_ratio_wsampling.pdf}
    \caption[The same as Figure \ref{fig:2d_ratio} but with 50 realisations where we remove 10\% of our central sample and replace these with satellites]{The same as Figure \ref{fig:2d_ratio} but with 50 realisations where we remove 10\% of our central sample and replace these with satellites. The additional realisations are plotted in the same colour with different transparency with the original distribution overplotted as before.}
    \label{fig:2d_ratio_realisations}
\end{figure}

Although a quantitative assessment of the effects of the group catalogue can only be made using a forward-model approach using mock catalogues we argue, based on the above, that the lack of signal reported in this section is more likely due to a lack of physical correlation between halo assembly history and kinematic misalignment measured up to 2.5\re.

\subsection{Summary}
In this section, we considered the visibility of cosmic web accretion and hence halo assembly onto central galaxies in MaNGA. We used the difference in global position angles measured for the stellar and H$\alpha$ velocity fields to classify if a galaxy is kinematically misaligned ($\Delta$PA > 30$^{\circ}$). This chapter is summarised as follows:
\begin{itemize}
\item We first correlated distances to cosmic web features such as nodes and filaments to the aligned and misaligned galaxy samples. We considered the theory that low-mass haloes embedded in filaments (or in close vicinity) find their accretion `stalled' as material moves preferentially towards larger sub-haloes along the filament. This would correspond to aligned central galaxies in low-mass haloes residing closer to filamentary structures. We find that kinematic misalignment holds little or no correlation with the vicinity to nodes or filaments once the effects of morphology, stellar mass and small scale density are considered, as shown in Figure \ref{fig:mh_cw}. 
\item We secondly correlated a proxy for halo age; the central stellar mass to total halo mass ratio, with kinematic misalignment. We explored the idea that large-scale tidal forces dictate the formation time-scales of low-mass haloes ($\lesssim$ 10$^{12.3} M_{\odot}$) which should be reflected both in the halo age but also the likelihood of on-going filamentary accretion being quenched. We found that the magnitude of kinematic misalignment held little or no relation to the proxy of the halo age, as shown in Table \ref{tab:chisq}. 
\item We finally considered the halo occupation distribution as a measure of halo age with older haloes providing more time for satellites to merge and hence decrease the magnitude of the HOD \citep[e.g.][]{zehavi2018}. We estimate the HODs using the background subtraction technique for the aligned and misaligned groups with application of stellar mass weightings between the samples. Regardless of the magnitude limit imposed, we find no statistically significant difference between the groups containing aligned and misaligned galaxies, as seen in Figure \ref{fig:hod_mpl6}. We note in this analysis we split at $\Delta$PA = 10$^{\circ}$ in order to construct a sample size large enough for comparison. While this difference is likely well above the expected average error in $\Delta$PA, internal processes may be erroneously included.
\end{itemize}

We note that the lack of correlation could be indicative that the role of `hot mode' accretion from the cooling of the hot halo may play a far larger role than `cold mode' accretion deriving from the cosmic web flows, even at lower halo masses. The ability of integral field spectroscopy to resolve positions of properties such as gas-phase metallicity and star formation rate histories with respect to the surrounding large-scale environment should shed light on the exact origin of misalignment in future MaNGA studies.

\section{Conclusions}
In this chapter we investigated the role of the cosmic web in dark matter halo assembly, through the use of galaxy kinematics. We demonstrated that the spin direction of disks for LTGs (lenticulars) preferentially orientate parallel (perpendicular) to the direction of the nearest filament. Conversely we found that kinematic misalignment in central galaxies showed little or no correlation with various observational measures of dark matter halo assembly (i.e. vicinity to morphological features in the cosmic web, sub-structure fraction, and HOD modelling).

The relationship between the cosmic web and galaxy observables is nuanced and isolating the impact from the driving factor of halo mass is difficult. Even with surveys such as MaNGA, providing detailed spatially resolved information for thousands of galaxies, tracing different assembly histories is tricky and \textit{finding the correct observational proxy is crucial}. A combination of hydrodynamical simulations and IFS observations remain an incredibly powerful method of understanding the galaxy-halo connection, especially in intuiting how kinematics can be effectively used to trace the impact of large scale structure. 

As highlighted in \S\ref{sec:halo_assembly_discussion}, one difficulty in finding the impact of recent accretion from the cosmic web could be due to the limitations in the field of view. As demonstrated in the next chapter (see Figure \ref{fig:beta_stack}), we find that the anisotropy of the cosmic web impacts orbits (and hence potentially kinematics) most significantly on the outskirts of the halo. At larger radii outside the direct influence of significant baryonic feedback (i.e. AGN), kinematics are more likely to be more unbiased tracers of cosmic web accretion (\red{reference}). Upcoming programs such as \red{MAGPI, WEAVE-APERTIF, HECTOR - not the right references here}, will enable future work relating the cosmic web through observations up to higher effective radii, and radio observations of cold gas (more likely to trace the cold phase gas accretion). In the next chapter we investigate the viability of considering the impact of cosmic web at even higher effective radii. Rather than using central galaxy kinematics, we consider whether the \textit{dynamics} of satellite galaxy orbits can be used to trace the anisotropy. 