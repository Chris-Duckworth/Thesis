% \epigraph{Father Dougal: Didn't you tell me once that Father Jack had a trial for Liverpool?
% \\ Father Ted: No... no, he was on trial, in Liverpool.}
\chapter{Introduction}
\vspace{-5in}
\includegraphics[height=2.0in]{thesis/latex/st_a_logo_.png}
\vspace{3in}

\label{ch:intro}
\section{Galaxy evolution}
\section{Angular momentum build up in galaxies}
Angular momentum is one of the key properties that quantifies a galaxy. Within the $\Lambda$ cold dark matter ($\Lambda$CDM) paradigm, galaxies form from the cooling and condensation of the initial gas cloud within dark matter haloes \citep{white1978, mo1998}. In the basic picture, the angular momentum content of the galaxy is inherited from the surrounding halo \citep[][]{fall1980}. In turn, this is acquired through tidal torques in the early growth phase from the large-scale structure \citep[e.g.][]{peebles1969, Doroshkevich1970}. If gravitational collapse proceeds unhindered, the initial gas cloud will form a stable rotating disc which eventually evolves into the late type galaxies (LTGs) we observe today \citep{white1978}. Since stars form from the rotating gas, the natural expectation is that they will inherit its dynamical characteristics often leading to coherent rotation between dark matter, gas and stars in both magnitude and direction. However, in the non-linear regime there is good reason to believe that the rotation of dark matter, gas and stars may decouple from each other as galaxies evolve up-to $z=0$. 

The evolution of a galaxy from initial collapse to today is seldom completed in isolation. By its very nature, structure formation in $\Lambda$CDM is hierarchical with haloes undergoing bottom-up assembly from mergers of lower mass progenitors. After turnaround, the angular momentum of the baryons in a galaxy can be driven dramatically away from the expectations of TTT through external processes such as interactions or mergers. How such interactions alter angular momentum depend on the magnitude, orientation and gas content of the merger. For example, gas rich mergers in general spin up galaxies whereas gas poor mergers are seen to spin down galaxies \citep[][]{lagos2017,lagos2018}.

Developments in spectrographs have led to the advent of integral field spectroscopy (IFS) which provides spatially resolved spectra for galaxies. Establishing work in the field has been the Spectrographic Areal Unit for Research on Optical Nebulae \citep[SAURON;][]{sauron} and ATLAS\textsuperscript{3D} \citep{atlas3d} surveys, which have focused on early type galaxies (ETGs) in the local Universe. IFS has enabled kinematic classification through a proxy for angular momentum based on the stellar kinematics up to one effective radius ($\mathrm{R_e}$). Termed $\mathrm{\lambda_{Re}}$, the measure enabled the clear distinction between slow and fast rotating ETGs \citep{emsellem2007, emsellem2011}. While there is still debate over whether 1$\mathrm{R_{e}}$ is large enough to fully encapsulate the complete kinematic morphology of a galaxy \citep{foster2013,arnold2014}, it has opened the door for understanding the relationship between optical morphology and angular momentum. 

IFS surveys for $\sim$1000 of galaxies across all optical morphologies are now a reality. For example, the Sydney-AAO  Multi-object  Integral  field  spectrograph  survey \citep[][]{croom2012, bryant2015} has mapped $\sim$3400 galaxies upto $z\sim0.12$ across a variety of environments. Even larger is the Mapping Nearby Galaxies at Apache Point \citep[MaNGA;][]{bundy2015, blanton2017} survey which will map $\sim$10000 galaxies in the local Universe ($z=0-0.15$). By design MaNGA will create a sample of near flat number density distribution across absolute $i$-band magnitude and stellar mass.

Recent studies in these surveys and also simulations have demonstrated the close interlink between stellar angular momentum, stellar mass and morphology suggesting that late types and early type fast rotators form a continuous sequence rather than from fundamentally different formation pathways \citep[][]{cortese2016, lagos2017, graham2018}. Remarkably, despite the highly non-linear processes involved, current cosmological surveys predict that the stellar angular momentum in rotationally supported galaxies at $z=0$ is still conserved from that of the dark matter halo \citep[e.g.][]{genel2015}. 

In the extended theory of TTT, the spin of galaxies embedded in the larger-scale environment of the cosmic web can be seen to align with the direction of filaments \citep[e.g.][]{pichon2011,codis2015, laigle2015}. Low mass discs can accrete material most efficiently when its spin vector is aligned with the direction along the filament. Conversely, higher mass haloes can be formed through mergers in the plane along the filament, leading to a perpendicular spin alignment with the large scale structure. 

\section{Large scale structure}
\section{Halo assembly bias}
The clustering of dark matter haloes is dependent on their constituent masses. This translates observationally into more massive and passive galaxies being more clustered than their lower mass counterparts. The dependence of halo clustering on properties outside of the driving factor of halo mass has been termed `\textit{halo assembly bias}'. Halo spin, concentration and formation time have been found, in simulations, to alter halo clustering, however observationally we are yet to determine whether these are independent or a secondary effect of density and hence halo mass. The motivation of this thesis lies therein. 