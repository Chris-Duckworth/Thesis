\chapter{Conclusions and Outlook}
\vspace{-5in}
\includegraphics[height=1.21in]{thesis/latex/headers/icons.pdf}
\vspace{3in}

\section{Chapter summaries}
In this thesis, the power of using large-scale integral field spectroscopic surveys and cosmological hydrodynamical simulations \textit{in tandem} was utilised to understand how galaxy kinematics can used as a tool to understand the galaxy-halo connection. We demonstrate the relationship of stellar and gas kinematics with morphology and the angular momentum content of the surrounding dark matter halo \citep{duckworth2020a}, with black hole growth and feedback \citep{duckworth2020b} and vicinity to different features of the cosmic web \citep{duckworth2019}. We also investigate how angular momentum is connected to the large-scale environment in which the galaxy evolves in, demonstrating that galaxies of different morphologies preferentially align with neighbouring filaments (Kraljic, Duckworth et al. in prep.), and that galaxy spin is modulated by vicinity to the nodes (density peaks) and filaments of the cosmic web (Duckworth et al. in prep.). We also demonstrate the potential for a novel application of the axisymmetric Jean's equations in order to recover the underlying dark matter dynamics with the cosmic web. We motivate individual galaxies as tracers for the underlying potential, which aide studies aiming to investigate the relationship between dark matter dynamics (and accretion) and the cosmic web. The key findings of this thesis from each Chapter is as follows:

\begin{itemize}
    \item \textbf{Chapter 2:} We investigate the key relationships of kinematic misalignment (offset rotation between stars and gas) with a galaxy's morphology, stellar angular momentum, gas content, and, spin of the host dark matter halo. We make use of 6000 galaxies from the MaNGA integral field spectroscopic survey to demonstrate that misalignment is much more prevalent in earlier type galaxies, lowered angular momentum and a lower gas mass (both independent of morphology). We reproduce these trends by creating mock MaNGA observations in the cosmological hydrodynamical simulation of IllustrisTNG, which in turn we use to demonstrate that misalignment can be correlated with lower spin dark matter haloes going back to $z=1$.
    
    \item \textbf{Chapter 3:} We investigate the temporal relationship between kinematic misaligned galaxies and black hole activity in IllustrisTNG. Using the sample of mock MaNGA observations, we demonstrate that low mass misaligned galaxies (identified at $z=0$) have had significantly boosted black hole luminosity, growth and greater feedback (injected into the gas) over the last 8 Gyrs. The epoch of peak energy injection from the black hole feedback is more recent for quenched misaligned galaxies (identified at $z=0$), in comparison to star forming misaligned galaxies. The different persistence timescales of kinematic misalignment versus black activity, makes correlation at a single timestep (e.g. $z=0$) difficult in observations.
    
    \item \textbf{Chapter 4:} We investigate how the magnitude and direction of stellar angular momentum of galaxies is modulated by their morphology, stellar mass, and cosmic web environment. We make use of 8000 MaNGA galaxies to determine $\mathrm{\lambda_R}$ (proxy for stellar angular momentum) decreases independently with stellar mass and earlier morphologies. High mass central galaxies (i.e. $\mathrm{M_{stel} > 10^{10.4}M_{\odot}}$) show a dependence on cosmic environment, having lower spins closer to both nodes and filaments. We also determine the three-dimensional alignment between the stellar spin direction of 4000 MaNGA galaxies and their neighbouring filamentary structure of the cosmic web. We demonstrate that the spin of pure LTGs preferentially aligns parallel with the direction of the neighbouring filament, whereas, lenticulars (S0s) preferentially align perpendicular (in both cases driven by the low end of the sample).
    
    \item \textbf{Chapter 5:} We investigate the relationship between kinematic misalignment and large-scale environment, in the context of dark matter halo assembly. We make use of 4000 galaxies in MaNGA to determine that misalignment (used here as a proxy for current gas accretion rate) does not correlate with various halo assembly measures (halo formation time, HOD modelling) and vicinity to cosmic web features at fixed stellar or halo mass. 
    
    \item \textbf{Chapter 6:} We motivate a novel application of the axisymmetric Jeans' equations to recover the three dimensional orbits of galaxies in their surrounding large-scale environment. The velocity anisotropy of dark matter orbits modulates as a function of cosmic environment, and here we demonstrate that so do the galaxies. We introduce a formalism for applying the Jeans' equations to reproduce the velocity anisotropy from stacked potentials of satellites galaxies, with a view to apply this to upcoming spectroscopic surveys such as DESI.
\end{itemize}

\section{Outlook}
This thesis demonstrated the power of using large-scale integral field spectroscopic surveys \textit{in tandem} with cosmological hydrodynamical simulations to understand the galaxy-halo connection. This represents an opportunity to make huge steps forward in building a holistic picture of the angular momentum content of galaxies in the local Universe. MaNGA and SAMI have demonstrated the potential of large-scale IFS surveys, unveiling the diversity of angular momentum content through stellar and gas kinematics, and relationship with fundamental properties such as stellar mass, morphology, and, black hole feedback. Despite this, the exact relationship of the cosmic web with galaxy kinematics is very much an open question. One problem is galaxy sampling, with the completeness and magnitude limit of current large area spectroscopic surveys (such as SDSS) limiting ability to recover all but the largest features of the cosmic web. A second is the typical radial coverage of integral field surveys; since we expect the impact of accretion to be most visible on the outskirts of galaxies, pushing observations to higher radii is crucial in de-convolving signal from internal processes. The future of both redshift and IFS surveys is, however, \textit{bright}. 

The Dark Energy Spectroscopic Instrument (DESI) will observe the whole sky, providing a galaxy sample in the local Universe an order of magnitude deeper than SDSS. This will play an integral part of understanding the relationship between the cosmic web and galaxies in the local Universe. In turn, several upcoming IFS surveys are set to go deeper in understanding the role of environment. For example, Hector is the next major dark-time instrument for the AAT and aims to obtain a low-redshift galaxy survey of 15,000 galaxies with IFS observations, with an aim for >70\% imaged out to 2 effective radii. Hector's spectrographs will also have higher spectral resolution than both MaNGA and SAMI, enabling measurements of higher order moments (h3, h4) to link the kinematic tracers to simulations of galaxy merger histories, and, have an more intrincate understand of the accretion history onto the galaxy. 

Another exciting possibility, is the combination of IFS surveys with radio observations to understand the cold gas phases, as a method of identifying fresh gas accretion, potentially from filamentary structure. The HI-MaNGA project; MaNGA galaxies coupled with follow-up radio observation from the Robert C. Bryd Green Bank Telescope (GBT) and archival ALFALFA data will cover up-to 70\% of the completed MaNGA survey \citet{himanga}. In addition, the future survey WEAVE-Apertif (combining radio observations from APERTIF with IFU observations from WEAVE) will yield $\sim10^4$ galaxies with redshifts, neutral gas content, morphologies, dynamics, and dynamical masses. An important distinction of WEAVE-Apertif lies in its HI selection of targets. 

Another pathway in making direct comparisons between observations and simulations, is to extend the comparison to IFS surveys to higher redshifts. One such example, is the upcoming Middle Ages Galaxy Properties with Integral Field Spectroscopy (MAGPI) survey that will observe stellar and gas kinematics for $\sim 60$ central galaxies (and 40+ satellites) at $z\sim0.25-0.35$. The target sample will cover a wide range of halo masses, and will have environmental information from the redshift survey GAMA to better understand the physical processes responsible for the rapid transformation of galaxies in this epoch. Pushing to even higher redshifts (and hence relying only on ionized gas to trace kinematics), various KMOS programs (e.g. KMOS Redshift One Survey \citep[KROSS;][]{stott2016}, KMOS Galaxy Evolution Survey (KGES; Tiley in prep.)) have investigated angular momentum relations for 100s of star forming galaxies at $z=1-2$. An interesting prospect is combining IFS surveys at different epochs \citep[e.g.][]{tiley2019} to understand the evolutionary history of individual galaxy populations over the last $\sim10$Gyrs.

A number of realistic hydrodynamical simulations are now well established, however, despite their large size ($\sim 100$Mpc$^3$) this is often too small to directly reproduce mock IFS surveys (when approaching $\sim 10000$ galaxies). Our mock MaNGA sample was made at the point in the survey where only $\sim$6000 galaxies had been observed, and finding more unique matches from IllustrisTNG-100 would have been increasingly difficult. The IllustrisTNG suite of simulations, introduced a 300Mpc$^3$ box to provide a significantly larger galaxy sample, albeit at the cost of an order of magnitude loss in resolution. This issue is particular difficult when trying to construct a mock sample of galaxies with $M_{stel} < 10^{9.5}$ where there are only a few hundred stellar particles. Moving forward, advances in software development and computing power will enable larger box sizes, without having to sacrifice particle resolution as much. One such project is \texttt{EAGLE-XL} which will simulate a 300Mpc$^{3}$ periodic cube at a similar resolution to IllustrisTNG-300 (i.e. an order of magnitude lower than their fiducial 100Mpc$^3$ counterparts). However the project is worth highlighting due to the potential of its \textit{up-scalability}. \texttt{EAGLE-XL} aims to build a new open-source simulation framework, \texttt{SWIFT}, which uses task-based parallelism in order to almost perfectly scale weakly with the number of particles unlike other models such as \texttt{GADGET} or \texttt{GIZMO}. Open-source and highly scalable simulation codes represent the best possibility of realising high resolution and realistic hydrodynamical simulations of vast cosmological scales. 