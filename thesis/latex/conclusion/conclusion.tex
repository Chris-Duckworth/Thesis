\chapter{Conclusions and Outlook}
\vspace{-5in}
\includegraphics[height=2.0in]{thesis/latex/st_a_logo_.png}
\vspace{3in}

In this thesis, the power of using large-scale integral field spectroscopic surveys and cosmological hydrodynamical simulations \textit{in tandem} was utilised to understand how galaxy kinematics can used as a tool to understand the galaxy-halo connection. 

We demonstrate the relationship of stellar and gas kinematics with morphology and the angular momentum content of the surrounding dark matter halo \citep{duckworth2020a}, with black hole growth and feedback \citep{duckworth2020b} and vicinity to different features of the cosmic web \citep{duckworth2019}. We also investigate how angular momentum is connected to the large-scale environment in which the galaxy evolves in, demonstrating that galaxies of different morphologies preferentially align with neighbouring filaments (Kraljic, Duckworth et al. in prep.), \red{and that galaxy spin is modulated by vicinity to the nodes (density peaks) of the cosmic web (Duckworth et al. in prep.)}. We also demonstrate the potential for a novel application of the axisymmetric Jean's equations in order to recover the underlying dark matter dynamics with the cosmic web. We motivate individual galaxies as tracers for the underlying potential, which aide studies aiming to investigate the relationship between dark matter dynamics (and accretion) and the cosmic web.